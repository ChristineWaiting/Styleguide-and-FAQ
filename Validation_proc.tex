\documentclass{article}
\usepackage[utf8]{inputenc}
\usepackage{amsmath}
\usepackage{hyperref}
\usepackage{framed}
\usepackage{listings}
\usepackage{color}

\begin{document}

\section*{Validation Procedure of QuantLets according to the Styleguide}
The proposed procedure should be followed sequentially. All steps are illustrated by examples included in the styleguide.

\subsection*{Instructions} %Enter instruction text here
\begin{enumerate}
	\item Check MetaInfo - \label{MetaInfo-ini}\hyperref[MetaInfo]{Click to see example}
	\begin{enumerate}
		\item Name of QuantLet
		\item Published in Book/ Paper
		\item Description - at least 10 words; should begin 			with 			the verb, e.g. “plots the time series…”
		\item Keywords - at least 5 words (the more the 				merrier)
		List of keywords \url{http://quantnet.wiwi.hu-berlin.de/index.php?p=searchResults&w=allkeywords&sort=f}
		\item \textit{Input (optional)}
		\item \textit{Output (optional)}
		\item Author
		\item Submitted
		\item Datafile
	\end{enumerate}
	\item Check Code (Align QuantLet Code to Style Guide 			standards) 
	\begin {enumerate}
		\item check wether it works properly
		\item check wether datafile is appropriate
		\item check wether the description is appropriate
		\item check wether plots\textbackslash pictures on 				the website are appropriate
		\item \label{formatR-ini}\hyperref[formatR]{to format with the package	FormatR}
			\begin{enumerate}
				\item library(FormatR)  --\textgreater 							 tidy\_source(file='output.R')
				\item to check whether all the lines of the 					code are written properly
				\item check that the code still works 							properly	
			\end{enumerate}
		\item delete unnecessary empty lines (1-2 						mostly, max. 4)
		\item \label{all-equal-ini}\hyperref[all-equal]{change all "\textless -" with "="}
		\item \label{align-assign-ini}\hyperref[align-assign]{Align all subsequent assignments by "="}
		\item \label{indentation-ini}\hyperref[indentation]{Check indentation}
	\end{enumerate}
\end{enumerate}

\subsection*{Textbook example of perfect Q's}
\begin{itemize}
	\item \url{http://quantnet.wiwi.hu-berlin.de/index.php?			p=show&id=3204}
	\item \url{http://quantnet.wiwi.hu-berlin.de/index.php?			p=show&id=3164}
\end{itemize}

\section*{Styleguide}

\subsection{\label{MetaInfo}\hyperref[MetaInfo-ini]{Complete and ordered MetaInfo-Header}}
\lstset{ %
numbers=left,
backgroundcolor = \color{white},
breaklines = true}
\begin{lstlisting}[frame=single]
# --------------------------------
# Name of QuantLet:
# --------------------------------
# Published in:
# --------------------------------
# Description:
# --------------------------------
# Keywords:
# --------------------------------
# Input:
# --------------------------------
# Output:
# --------------------------------
# Author:
# --------------------------------
# Submitted:
# --------------------------------
# Datafile:
# --------------------------------
\end{lstlisting}

\subsection{\label{formatR} \hyperref[formatR-ini]{How to use formatR}}
\lstset{ %
numbers=left,
backgroundcolor = \color{white},
breaklines = true}
\begin{lstlisting}[frame=single]
# Cleaning up the source in an R script file "input.R", Indentation is set 
# to two space characters. Maximum line width is 80 characters. 
# The formatted code is written into a new script file "output.R"

tidy_source(source = "input.R", indent = 2, width.cutoff = 
80, file = "output.R")

# similar to the previous example, but using the clipboard 
# instead of an input file

tidy_source(indent = 2, width.cutoff = 80, file = "output.R")

# when omitting function parameters the defaults indent = 4 and 
# width.cutoff = 80 are being used. For simplicity, we recommend these for use on Quantnet.

tidy_source(file = "output.R")

\end{lstlisting}

\subsection{\label{all-equal}\hyperref[all-equal-ini]{Change all "\textless-" with "="}}
\lstset{ %
numbers=left,
backgroundcolor = \color{white},
breaklines = true}
\begin{lstlisting}[frame=single]
#BAD
foo <- 5.0
bar <- function(x) {
	return x^2
}

#GOOD
foo = 5.0
bar = function(x) {
	return x^2
}

\end{lstlisting}

\subsection{\label{align-assign}\hyperref[align-assign-ini]{Align assignments in subsequent lines by "="}}
\lstset{ %
numbers=left,
backgroundcolor = \color{white},
breaklines = true}
\begin{lstlisting}[frame=single]
foo	= 5.0
foobar 	= 7.0
bar 	= 8.0
\end{lstlisting}

\subsection{\label{indentation}\hyperref[indentation-ini]{Set four space characters or a single tab per indentation level}}
\lstset{ %
numbers=left,
backgroundcolor = \color{white},
breaklines = true}
\begin{lstlisting}[frame=single]
while (i < n){
    i = i + 1
}
\end{lstlisting}

\end{document}
